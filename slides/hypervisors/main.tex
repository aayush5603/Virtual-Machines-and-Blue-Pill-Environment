\section{Proxmox Virtual Environment}

\begin{frame}[allowframebreaks]{Proxmox Virtual Environment - PVE}
    \begin{columns}
        \begin{column}{0.6\linewidth}
            \begin{itemize}
                \item Debian-based free open-source virtualization management software
                \item Uses Kernel-based Virtual Machine (KVM) Hypervisor
                \item Support for clustering multiple server nodes
                \item Web-based management interface $\rightarrow$ easy to use and many tutorials online
                \item Installation very easy and similar to normal Linux distros
            \end{itemize}
        \end{column}
        \begin{column}{0.4\linewidth}
            \begin{figure}
                \centering
                \includegraphics[width=0.6\linewidth]{proxmox-logo.png}\\
                \cite{proxmox-logo}
            \end{figure}
        \end{column}
    \end{columns}
\end{frame}


\begin{frame}{Kernel-based Virtual Machine - KVM}
    \begin{itemize}
        \item Open-source virtualizing technology
        \item Integrated in Linux since version 2.6.20 (2007)
        \item KVM converts Linux to a bare-metal Hypervisor
        \item Every VM is managed as a regular Linux process
        \item Requires hardware virtualization support (AMD-V or Intel-VT-x)
    \end{itemize}
\end{frame}

\begin{frame}[noframenumbering,allowframebreaks]{Sources}
    \printbibliography[title = {Sources}, heading = none]
\end{frame}